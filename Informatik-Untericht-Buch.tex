\documentclass[12pt]{article}

\title{Informatik Untericht Buch}
\author{Lukas Scheifele}
\date{März 2023}

\renewcommand{\contentsname}{Inhaltsverzeichnis}

\begin{document}
\maketitle
\clearpage
\tableofcontents
\clearpage


\section{Die Geschichte und Anfang von Informatik}
\subsection{Schriffrierung}

Der Lehrer schmeist sein Buch einem zufälligen Schüler zu. Er sagt, dass er ihm Seitenzahlen nennen werde und dass er die Seiten des Buches aufschlagen und sich pro Seite das erste Wort notieren solle.
Die Seitenzahlen werden vom Lehrer diktiert, während der Schüler die Seiten durchblättert, um sich die Worte herauszufinden und zu notieren.
Nun nennt der Schüler seine Worte die nur er kennt. Für allen anderen Schülern wissen nicht was die Nachricht ist.

Dies ist Buch-Schriffrierung, eine von vielen Schriffrierung Methoden.
Willkommen im Informatik Untericht.

\subsubsection{Der Nachfolger der Schriffrierung: Verschlüssung}
Während Schriffrierung heutzutage einfach von Computer geknackt werden kann, in dem jegliche Kombination ausbrobiert wird, kann moderne Verschlüssung wie SHA-256 nicht in kürzeste Zeit geknackt werden.
Es ist jedoch wichtig anzumerken, dass es veraltete Verschlüssungsmethoden gibt die wie Schriffrierung nicht sicher sind. Außerdem kann es kommen, dass die Verschlüssungsmethoden die heutzutage sicher sind, 
unsicher werden weil die Computer weiterentwickelt werden und somit schneller die möglichen Kombinationen ausprobieren können.


\subsection{Morse Code}
Der Morsecode ist ein Kodierungsverfahren, das von Samuel Morse in den 1830er Jahren für den Telegrafen entwickelt wurde. 
Der Telegraf war ein Gerät, das elektrische Signale nutzte, um Nachrichten über große Entfernungen mithilfe von Drähte zu übertragen.


Beim Morsealphabet werden Buchstaben und Zahlen durch eine Kombination aus Punkten und Strichen dargestellt. 
Ein Punkt steht für ein kurzes Signal, ein Strich für ein längeres Signal.

Jeder Buchstabe und jede Zahl wird durch eine einzigartige Kombination von Punkten und Strichen dargestellt. 
So wird beispielsweise der Buchstabe "A" durch einen Punkt gefolgt von einem Bindestrich dargestellt, während der Buchstabe "B" durch einen Bindestrich gefolgt von drei Punkten dargestellt wird.


\begin{table}[h]
\centering
\begin{tabular}{|c|c|}
\toprule
\textbf{Letter} & \textbf{Morse Code} \
\midrule
A & .- \
B & -... \
C & -.-. \
D & -.. \
E & . \
F & ..-. \
G & --. \
H & .... \
I & .. \
J & .--- \
K & -.- \
L & .-.. \
M & -- \
N & -. \
O & --- \
P & .--. \
Q & --.- \
R & .-. \
S & ... \
T & - \
U & ..- \
V & ...- \
W & .-- \
X & -..- \
Y & -.-- \
Z & --.. \
0 & ----- \
1 & .---- \
2 & ..--- \
3 & ...-- \
4 & ....- \
5 & ..... \
6 & -.... \
7 & --... \
8 & ---.. \
9 & ----. \
\bottomrule
\end{tabular}
\caption{Morse Code Alphabet}
\end{table}


Der Morsecode ist eine einfache und effiziente Kommunikationsmethode, die seit über
seit über 150 Jahren verwendet wird. Trotz der Entwicklung fortschrittlicherer Kommunikationstechnologien
bleibt sie ein wichtiger Bestandteil der Geschichte und wird weiterhin im Militär und in FAX Geräten verwendet.


\subsection{Erfinder des ersten programmierbaren Computer: Konrad Zuse}
Konrad Zuse war ein deutscher Ingenieur und Erfinder, der als der Erfinder des ersten programmierbaren Computers gilt. Er wurde am 22. Juni 1910 in Berlin geboren und wuchs in Hoyerswerda auf. 
Schon früh zeigte er Interesse an Technik und baute in seiner Freizeit elektrische Schaltungen und Radios.
Nach dem Abitur studierte Zuse Bauingenieurwesen, jedoch brach er das Studium ab, um sich vollständig der Technik zu widmen. 
Im Jahr 1935 erfand er eine Maschine, die automatisch Berechnungen durchführen konnte, indem sie Lochkarten verarbeitete. Diese Maschine nannte er Z1 und sie gilt als der erste funktionsfähige Computer der Welt.
Während des Zweiten Weltkriegs arbeitete Zuse weiterhin an seinen Computern und entwickelte den Z2 und den Z3, die noch fortschrittlicher waren als die Z1. 
Nach dem Krieg gründete er sein eigenes Unternehmen, die Zuse KG, und begann mit der Entwicklung von Hochgeschwindigkeitscomputern.

Konrad Zuse starb am 18. Dezember 1995 in Hünfeld, Deutschland. Seine Beiträge zur Informatik haben die Welt nachhaltig beeinflusst und werden noch heute in vielen Bereichen eingesetzt.

//TODO: Dieses Kapitel


\section{Hardware}
\subsection{Peripherie Geräte}
\subsection{Eingabegeräte}
\subsection{Ausgabegeräte}
\subsection{Schnittstellen von PS/2 bis HDMI}
\subsection{Computer Hardware Komponenten}
\subsubsection{Mainboard}
\subsubsection{CPU}
\subsubsection{RAM}
\subsubsection{GPU}
\subsubsection{Datenspeicher: HDD, SSHD, SSD, M.2 SSD, ROM}
\subsubsection{Netzteil und die Stromversorgung der Komponenten}
\subsubsection{Kommunikation zwischen Mainboard und Gehäuse}

\section{Sicherheit, Datenschutz und Datensicherung}

\section{Grundlägige Nutzung eines Computers}
\subsection{Internet und die Richtigkeit von Informationen }
\subsection{Webbrowser: Suchmaschiene, recherschieren}
\subsection{LibreOffice / Microsoft Office}
\subsection{Erstellen von Dokumenten}
\subsection{Erstellen von Präsentationen}
\subsection{Tabellenkalkulation}
\subsection{Pixelgrafik}
\subsection{Vector Grafik}

\section{Logik}

\section{Softwareentwicklung}
\subsection{Anweisungen und Programmablauf}
\subsection{Datentypen}
\subsection{Operatoren}


\begin{align}
\end{align}

\end{document}